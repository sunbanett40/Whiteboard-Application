%%%%%%%%%%%%%%%%%%%%%%%%%%%%%%%%%%
% EEE Report Template
% University of Southampton
%
% authors: George Brown (gb4g15)
%          Rhys Thomas  (rt8g15)
%
% edited : 2016/02/05
%%%%%%%%%%%%%%%%%%%%%%%%%%%%%%%%%%

\documentclass[10pt]{article}

%%%%%%%%%%%%%%%%%%%%%%%%%%%%%%%%%%
% PACKAGES
%%%%%%%%%%%%%%%%%%%%%%%%%%%%%%%%%%
\usepackage{multicol} % multiple columns
\usepackage{fancyhdr} % page number in bottom right
\usepackage{graphicx} % images
\usepackage{float} % float image in columns, gives [H]
\usepackage{amsmath, amssymb} % maths symbols / environments
\usepackage{times} % times font instead of computer modern
\usepackage{IEEEtrantools} % et al referencing
\usepackage{booktabs} % nice replacements for \hline in tables
\usepackage{calc} % calculating textwidth-2cm etc.
\usepackage[none]{hyphenat} % no hyphenation
\usepackage{geometry} % Page Margins
\usepackage{hyperref} % PDF metadata setup.
\usepackage{mathptmx} % Use times font face for maths
\usepackage{caption} % Tighter caption control.

\usepackage{lipsum} % Temporary to generate body text

%%%%%%%%%%%%%%%%%%%%%%%%%%%%%%%%%%
% TITLE AND AUTHOR
%%%%%%%%%%%%%%%%%%%%%%%%%%%%%%%%%%
% This info is reused in places, so input it here and it will be updated globally.
\newcommand{\docTitle}{Whiteboard Chat Applictaion with Half-Duplex Communication}
\newcommand{\docAuthor}{Joesph Butterworth}

% Put the metadata in the PDF output.
\hypersetup{
    unicode=true,
    pdftitle={\docTitle{}},
    pdfauthor={\docAuthor{}}
}

%%%%%%%%%%%%%%%%%%%%%%%%%%%%%%%%%%
% FORMATTING REQUIREMENTS
%%%%%%%%%%%%%%%%%%%%%%%%%%%%%%%%%%
\geometry{top=2.5cm, bottom=2.5cm, left=2cm, right=2cm}
\linespread{1.05} % 1.05x line spacing.
\setlength{\columnsep}{0.7cm} % 0.7cm column spacing.
\setlength{\multicolsep}{0cm}
\setlength{\parskip}{6pt} % 6pt skip between paragraphs
\setlength{\parindent}{0pt}
\newcommand{\figsquish}{\vspace{-5mm}} % Hack to fix poor figure spacing due to [H]

% Captions
% Table captions go above, 6pt space, small (9pt) roman font, roman numeral counting.
\captionsetup[table]{position=above, skip=6pt, font={small, rm}}
\renewcommand\thetable{\Roman{table}}
% Figure captions go below, 5pt space, small (9pt) roman font.
\captionsetup[figure]{position=below, skip=5pt, font={small, rm}}

%%%%%%%%%%%%%%%%%%%%%%%%%%%%%%%%%%
% SECTION REQUIREMENTS
%%%%%%%%%%%%%%%%%%%%%%%%%%%%%%%%%%
\usepackage{titlesec}
\titlelabel{\thetitle.\hspace{0.5cm}} % Dot between number and title on sections.
% Format is 10pt, so \large = 12pt, \normalsize=10pt
\titleformat*{\section}{\large\bfseries}
\titlespacing*{\section}{0cm}{4pt}{4pt} % 6pt from \parindent
\titleformat*{\subsection}{\normalsize\bfseries}
\titlespacing*{\subsection}{0cm}{0pt}{0pt} % 6pt from \parindent

% Set up footer.
\pagestyle{fancy}
\fancyhf{}
\renewcommand{\headrulewidth}{0pt}
\rfoot{\thepage} % page number, bottom right of page

%%%%%%%%%%%%%%%%%%%%%%%%%%%%%%%%%%
% DOCUMENT BEGIN
%%%%%%%%%%%%%%%%%%%%%%%%%%%%%%%%%%

\begin{document}

% Pull in the IEEE referencing setup stuff.
\bstctlcite{IEEEexample:BSTcontrol}

%%%%%%%%%%%%%%%%%%%%%%%%%%%%%%%%%%
% HEADER
%%%%%%%%%%%%%%%%%%%%%%%%%%%%%%%%%%
{
    \centering
    % Use size 28 font. 1.05x gives 29.4pt line spacing.
    \fontsize{28pt}{29.4pt} \selectfont
    \docTitle\\
    \vspace{25pt}
    % Name block.
    \fontsize{11pt}{11.55pt}\selectfont
    \docAuthor\\
    \fontsize{10pt}{10.5pt}\selectfont
    \textit{jdb1g20@soton.ac.uk} \\ % add your email address
    \textit{MEng Electrical and Electronic Engineering} \\ 
    \textit{Personal Tutor: Tracy Melvin} \\ % add your personal tutor
}
\vspace{25pt}

%%%%%%%%%%%%%%%%%%%%%%%%%%%%%%%%%%
% ABSTRACT
%%%%%%%%%%%%%%%%%%%%%%%%%%%%%%%%%%
{
\setlength{\tabcolsep}{0cm} % No additional column spacing since we've set it strictly.
\centering
\begin{tabular}{p{2cm}p{\textwidth-2cm}}
    Abstract: &
    
\end{tabular}  
}
\vspace{25pt}

%%%%%%%%%%%%%%%%%%%%%%%%%%%%%%%%%%
% BODY
%%%%%%%%%%%%%%%%%%%%%%%%%%%%%%%%%%
\begin{multicols*}{2}

\section{Introduction}

\section{Theory}
\subsection{QPainter}
How will the send-window allow users to draw diagrams? How will it display
diagrams as they are being drawn, and how will it retain these diagrams so
that they don’t disappear when the window is repainted?

\subsection{Threads and Mutexes}
text

How will you use threads to send and receive these packets, while the rest of
the application keeps running? How will you use mutexes to make any relevant
collections “thread-safe”?

\subsection{Serialisation}
How will you serialize these commands into packets to be sent from the send
window?

How will you convert binary packets into a stream of 1’s and 0’s? How will you
transmit this stream in a reliable way? For example, you may need to signal
when a bit is ready to be read, and when the receive window has finished
reading the current bit.

\subsection{Simplex, Half-Duplex, and Duplex Communication}
How will you represent the drawing commands so that they can be sent to the
other user whilst they are being drawn?

How will you receive and buffer packets at the other end? How will you
deserialize them? How will you draw them on the receive window? How will
you retain the currently received diagram so that when the window is
repainted the diagram isn’t lost?

\section{Implementation}
\subsection{The GUI and Receive Window}
text

\subsection{The Serialise and Deserialise Drawing Commands}
text

\subsection{The Send and Receive Threads}
text

\subsection{Communication Protocol using Booleans}
text

\section{Final Application}

\section{Discussion}

\section{Conclusion}


%%%%%%%%%%%%%%%%%%%%%%%%%%%%%%%%%%
% BIBLIOGRAPHY
%%%%%%%%%%%%%%%%%%%%%%%%%%%%%%%%%%
\nocite{*} % show all references even without citation
% to cite use "bla bla"~\cite{ref_label} -> "bla bla" [1]
\bibliographystyle{IEEEtran}
% IEEEabrv abbreviates journal titles in accordance to IEEE standards 
\bibliography{IEEEabrv,mybib}

\end{multicols*}
\end{document}
